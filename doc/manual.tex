% Every Latex document starts with a documentclass command
\documentclass[a4paper, 11pt]{article}

% Load some packages
\usepackage{graphicx} % This allows you to put figures in
\usepackage{natbib}   % This allows for relatively pain-free reference lists
\usepackage[left=2cm,top=3cm,right=2cm]{geometry} % The way I like the margins

\newcommand{\dnest}{{\tt DNest3}}

% This helps with figure placement
\renewcommand{\topfraction}{0.85}
\renewcommand{\textfraction}{0.1}
\parindent=0cm

% Set values so you can have a title
\title{{\tt DNest3} Manual}
\author{Brendon J. Brewer}
\date{\today}

% Document starts here
\begin{document}

% Actually put the title in
\maketitle

\section{Introduction}

\dnest~is a multi-threaded C++ implementation of Diffusive Nested Sampling, a
powerful Markov Chain Monte Carlo (MCMC) algorithm that is primarily
useful for solving Bayesian Inference problems. For information about how the
algorithm works, please see the paper, available as a preprint at the following
URL:

\begin{center}
{\tt http://arxiv.org/abs/0912.238}
\end{center}

If you find \dnest~useful, please feel free to cite the paper. \dnest~is
released under the terms of the GNU General Public Licence, version 3.
In the following I will assume that you have read the paper and have a
reasonable understanding of how the algorithm works.

\section{Installation}
Please note that I have only ever compiled \dnest~on GNU/Linux,
using the GNU C++ compiler ({\tt g++}). However, I expect it to be
straightforward to compile it on any other Unix-like operating system such as Mac
OS X or FreeBSD. It should be possible, albeit probably more tricky,
to compile \dnest~on Microsoft Windows. If you want to try,
the MinGW compiler is
probably your best bet. I'd be interested to hear from anyone who
has tried this. I would also be interested to hear if anyone has
compiled \dnest~with a different compiler, such as the Intel C++
compiler.

\subsection{Dependencies}
\dnest~depends on some other software that you will need to have installed on
your system. These should be quite easy to obtain. I highly
recommend obtaining these programs from your operating system's package manager,
rather than installing them from source. Note that some OSs split up
packages into binaries and "development" packages, with a -dev (e.g.
in Debian or Ubuntu) or -devel suffix (e.g. Fedora). You will need
both. Here's a list of dependencies:

\begin{itemize}
\item The GNU Scientific Library (GSL)
\item Python 2, along with NumPy and matplotlib
\item The thread and system parts of the Boost C++ library.
It may be easier to just install all of Boost.
\end{itemize}

On Ubuntu, I can install these packages with the following commands:
\begin{verbatim}
sudo apt-get install libgsl0-dev
sudo apt-get install python-numpy python-matplotlib
sudo apt-get install libboost-thread-dev libboost-system-dev
\end{verbatim}

\dnest~uses GSL's random number generator. Python, NumPy and
matplotlib are used for the postprocessing scripts and for
plotting. Boost is used for parallelization. While random number generation and
multithreading is supported
in the recent C++11 standard, I have not ported \dnest~to C++11 and have no
intention of doing so in the immediate future.

\subsection{Compiling}
Go into the \dnest~directory and type {\tt make}. Everything should compile. To
verify that everything worked, go into the {\tt Examples/SpikeSlab} directory
and see if a file called {\tt main} has been created. If so, everything worked
correctly.

\section{Running DNest}
As part of the compilation process, several examples will have been compiled.
The executable files are per-example, there is no global executable.
To run the {\tt SpikeSlab} example, go into the {\tt Examples/Spikeslab}
directory and execute {\tt main}. You should see some output that looks like
that shown in Figure~\ref{fig:output}.
Note that you must terminate the process manually by pressing Ctrl+C
or in another way. If you do not terminate the process yourself, it will run
forever. There is no hard and fast rule for how long you should run \dnest.
It depends on the difficulty of your problem, and the number of samples you
want.

\begin{figure}[h!]
\begin{verbatim}
$ ./main
# Using 1 thread.
# Target compression factor between levels = 2.7182818284590451
# Seeding random number generator with 1402801184.
# Thread 1: Generated 1 particles from the prior.
# Creating level 1 with logL = -51.65058877.
# Creating level 2 with logL = -37.67550971.
# Creating level 3 with logL = -28.69668897.
# Creating level 4 with logL = -21.61338509.
# Creating level 5 with logL = -15.92331209.
# Creating level 6 with logL = -11.291182.
# Saving a particle to disk. N = 1.
# Creating level 7 with logL = -7.301954778.
# Creating level 8 with logL = -4.116771404.
# Creating level 9 with logL = -1.19393518.
# Creating level 10 with logL = 1.620392784.
# Saving a particle to disk. N = 2.
\end{verbatim}
\caption{Example \dnest~output.\label{fig:output}}
\end{figure}

The executable {\tt main} is responsible for the exploration part of the
Diffusive Nested Sampling algorithm. It creates three output files,
{\tt sample.txt}, {\tt sample\_info.txt}, {\tt levels.txt}.

The first output
file, {\tt sample.txt}, contains a sampling of parameter values that
represents the {\it mixture of constrained priors}, i.e. {\bf not} the
posterior distribution. Each line of {\tt sample.txt} represents a point in
parameter space. Each time a point is saved to {\tt sample.txt}, \dnest~prints
the message ``Saving a particle to disk. N = ...''. To actually get posterior
samples, you need to run the {\tt showresults.py} script.

The second output file, {\tt sample\_info.txt}, should have the same number of
lines as {\tt sample.txt}. It contains metadata about the samples. The first
column is the index $j$, which tells us the level the particle was in when it
was saved. The second column is the log-likelihood, and the third column is
the likelihood tiebreaker. The final column tells us which thread the particle
belonged to.

The third output file, {\tt levels.txt}, contains information about the levels
that were built during the run.

\section{Postprocessing with showresults.py}
After {\tt main} has been running for
a while, you should run the postprocessing script
by executing {\tt python showresults.py}. This script loads the output files,
plots some graphs, and creates two new output files, {\tt weights.txt} and
{\tt posterior\_samples.txt}.


\section{Command line options}
\dnest~executables allow for several command line options. These options are
quite important. You can view the available options by executing
{\tt ./main -h}. Here is the list:

\begin{verbatim}
DNest3 Command Line Options:
-h: display this message
-l <filename>: load level structure from the specified file.
-o <filename>: load DNest3 options from the specified file. Default=OPTIONS
-s <seed>: seed the random number generator with the specified value.
           If unspecified, the system time is used.
-d <filename>: Load data from the specified file, if required.
-c <value>: Specify a compression value (between levels) other than e.
-t <num_threads>: run on the specified number of threads. Default=1.
\end{verbatim}


\section{The OPTIONS file}


%    <h3>Compiling<br>
%    </h3>
%    To compile DNest3, simply run<br>
%    <br>
%    <span style="font-family: monospace;">make</span><br>
%    <br>
%    in the root DNest3 directory. This will create <span
%      style="font-family: monospace;"></span>a static library <span
%      style="font-family: monospace;">libdnest3.a</span> in the current
%    directory. If you like, feel free to copy these to some other
%    location on your system, for example <span style="font-family:
%      monospace;">/usr/local/lib</span>. It will also compile the
%    examples in the <span style="font-family: monospace;">Examples/</span>
%    directory, creating executable files called <span
%      style="font-family: monospace;">main</span> that you can use to
%    run the examples.<br>
%    <h3>Running the Examples</h3>
%    Two examples are provided with DNest3. I will now explain how DNest3
%    is used through the first example, <span style="font-style:
%      italic;">SpikeSlab</span>.<br>
%    <h4>Example 1: SpikeSlab</h4>
%    The first example, called <span style="font-style: italic;">SpikeSlab</span>,
%    is the demo problem from our <a
%      href="http://arxiv.org/abs/0912.2380">paper</a>, and is a slight
%    modification of one of the examples in <a
%href="http://citeseerx.ist.psu.edu/viewdoc/download?doi=10.1.1.117.5542&amp;rep=rep1&amp;type=pdf">John










%      Skilling's Nested Sampling Paper</a>. It is a problem with 20
%    unknown parameters, each with a uniform prior between -0.5 and 0.5.
%    The priors for all of the parameters are independent. The likelihood
%    function is a mixture of two Gaussians: one is a wide "slab" and the
%    other is a narrow "spike". The slab is centered at (0, 0, ..., 0)
%    with a width of 0.1 in each dimension. The spike is centered at
%    (0.031, 0.031, ..., 0.031) and has a width of 0.01 in each
%    dimension. This problem is challenging for all sampling algorithms
%    that are not variants of Nested Sampling.<br>
%    <br>
%    To run DNest3 on the <span style="font-style: italic;">SpikeSlab</span>
%    example, simply compile DNest3, then enter the SpikeSlab directory
%    and run <span style="font-family: monospace;">main</span>. <br>
%    <br>
%    <code><span style="font-family: monospace;">make<br>
%        cd Examples/SpikeSlab<br>
%        ./main</span></code><br style="font-family: monospace;">
%    <br>
%    You should see a bunch of output that looks like this:<br>
%    <samp><br>
%      # Using 1 thread.<br>
%      # Seeding random number generator with -1337625108.<br>
%      # Generating 3 particles from the prior...done.<br>
%      # Creating level 1 with logL = -51.91599704.<br>
%      # Creating level 2 with logL = -38.68866085.<br>
%      # Creating level 3 with logL = -29.30928837.<br>
%      # Creating level 4 with logL = -22.44227047.<br>
%      # Creating level 5 with logL = -16.947356.<br>
%      # Creating level 6 with logL = -12.19901323.<br>
%      # Saving a particle to disk. N = 1.<br>
%      # Creating level 7 with logL = -8.14478528.<br>
%      # Creating level 8 with logL = -4.640384951.<br>
%      # Creating level 9 with logL = -1.486986762.<br>
%      # Creating level 10 with logL = 1.317628374.<br>
%      # Saving a particle to disk. N = 2.<br>
%      # Creating level 11 with logL = 4.137860699.<br>
%      # Creating level 12 with logL = 6.379241865.<br>
%      # Creating level 13 with logL = 8.447578631.<br>
%      # Saving a particle to disk. N = 3.<br>
%      # Creating level 14 with logL = 10.09126447.<br>
%      # Creating level 15 with logL = 11.68745251.<span
%        style="font-family: monospace;"></span><br style="font-family:
%        monospace;">
%    </samp><br>
%    and so on. By default, this process will continue forever (this can
%    be overridden by using different OPTIONS). At some point, you should
%    kill it with Ctrl-C. Then, it is time to examine and process the
%    output. The executable itself creates the following output files:<br>
%    <br style="font-family: monospace;">
%    <span style="font-family: monospace;">levels.txt</span><br
%      style="font-family: monospace;">
%    <span style="font-family: monospace;">sample.txt</span><br
%      style="font-family: monospace;">
%    <span style="font-family: monospace;">sample_info.txt</span><br
%      style="font-family: monospace;">
%    <br>
%    Of these, <span style="font-family: monospace;">sample.txt</span>
%    is the most important file, as it contains the samples. Each line
%    corresponds to a sample, a point in the parameter space. However,
%    the samples in <span style="font-family: monospace;">sample.txt</span>
%    are <span style="font-style: italic;">not</span> posterior samples.
%    Instead, they are samples from the mixture distribution that DNest3
%    actually explores. To actually get posterior samples, you need to
%    run the post-processing script <span style="font-family:
%      monospace;">showresults.py</span>. This script produces useful
%    plots of the sort shown in the <a
%      href="http://arxiv.org/abs/0912.2380">paper</a>, and also creates
%    the following additional output files that will probably be of more
%    interest to you:<br>
%    <br>
%    <span style="font-family: monospace;">weights.txt</span><br
%      style="font-family: monospace;">
%    <span style="font-family: monospace;">posterior_sample.txt</span><span
%      style="font-family: monospace;"></span><br style="font-family:
%      monospace;">
%    <br>
%    The file <span style="font-family: monospace;">weights.txt</span>
%    contains importance weights for the samples in <span
%      style="font-family: monospace;">sample.txt</span>. Alternatively,
%    if you are not used to dealing with non-equally-weighted samples,
%    the file <span style="font-family: monospace;">posterior_sample.txt</span>
%    is for you!<br>
%    <br>
%    <span style="color: rgb(0, 0, 153);">Note: You can also run <span
%        style="font-family: monospace;">showresults.py</span> while the
%      main process is still running. In fact, this is recommended in any
%      non-trivial problem for monitoring the algorithm's progress. The
%      only issue is that the main process may write to the output files
%      while <span style="font-family: monospace;">showresults.py</span>
%      is trying to load them. This may case an "ERROR: Size mismatch"
%      message to be displayed. The workaround is to suspend the main
%      process with Ctrl-Z, run <span style="font-family: monospace;">showresults.py</span>
%      and then resume the main process with <span style="font-family:
%        monospace;">fg</span>.</span><span style="font-weight: bold;"><br>
%      <br>
%    </span>
%    <h3>Command-Line Options<br>
%    </h3>
%    Run <span style="color: rgb(0, 0, 153);"><span style="font-family:
%        monospace;">./main -h </span></span>to see a list of available
%    command-line options.<br>
%    <h3>TODO: Tuning the Numerical Parameters in 'OPTIONS'</h3>
%    <h3>TODO: Implementing your own Models</h3>
%    <h3>TODO: Tips and Tricks</h3>
%    <span style="font-family: serif;"> </span>
%    <h3>Other Recommended Samplers</h3>
%    While DNest3 is a very effective sampler and works on a wide range
%    of problems (including many problems with multi-modal posterior
%    distributions or strong correlations), there are a few situations
%    where I would recommend alternative methods. The primary situation
%    in which DNest3 is not recommended is where the calls to the
%    likelihood function are expensive, yet the number of parameters is
%    small to moderate.<br>
%    <br>
%    For these problems, I would recommend <a
%      href="http://github.com/dfm/emcee">emcee</a> in the case where
%    strong correlations may be present but multimodality is unlikely, or
%    <a href="http://ccpforge.cse.rl.ac.uk/gf/project/multinest/">MultiNest</a>
%    if multimodality is expected or if the value of the evidence
%    integral is required. These samplers also have the advantage of
%    requring less programming from the user. For example, emcee only
%    requires a likelihood function, whereas DNest3 requires that you
%    implement a likelihood function, a function to generate random
%    models from the prior, and a function for generating
%    Metropolis-Hastings proposals.<br>




\section{Running the examples}


\section{Writing your own models}



\end{document}

